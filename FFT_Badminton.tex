\documentclass{article}


%defining all packages
\usepackage[utf8]{inputenc}
\usepackage{amsmath}
\usepackage{graphicx}
\usepackage{amsfonts}
\usepackage{algorithm}
\usepackage{adjustbox}
\usepackage{listings}
\usepackage{color}
\usepackage{algpseudocode}
\graphicspath{{/FFTBadminton}}
\graphicspath{{/home/soham/Uni/Quant/FFTBadminton/}}

%end of packages


%title page
\title{Application of FFT with Badminton}

\author{Soham Deshpande}
\date{November 2022}

\begin{document}

\maketitle
\clearpage
\tableofcontents
\clearpage
%end of title page

\section{Introduction}

After playing badminton for 10 years, a tribute was much needed. The best way
to combine my passion of badminton and mathematical computation led me to explore
the use of Fourier Analysis in Image Processing. I have taken an image of the badminton
racket I am currently using and have used Matlab to return the fourier transform
of it. The following pages will contain a quick explanation of the maths, code
and the result.
\clearpage


\section{Fourier Transform}

The Fourier tranform extends the fourier series so that it can be applied to non-periodic
functions, which helps us view any function in terms of sum of sinusoids.
\\
The Fourier tranform of a function $f(x)$ is given by:
\begin{equation}
    \int_{\infty}^{-\infty}F(k)e^{2\pi ikx}dk
\end{equation}

We can use this defintion and then further extend it to be used in a 2-D Fourier
Transform. The formula below defines the discrete the Fourier Tranform Y of
an $m$ by $n$ matrix X:

\begin{equation}
    Y_{p+1,q+1} = \sum^{m-1}_{j=0}\sum_{k=0}^{n-1}\omega^{jp}_m\omega_{n}^{kq}X_{j+1,k+1}

\end{equation}

$\omega_m$ and $\omega_n$ are the complex roots of unity. These can be defined as:
\\
$\omega_m = e^{-2 \pi i/m}$
\\$\omega_n = e^{-2 \pi i/n}$

$i$ is the imaginary number. The formula above shifts the indicies for X and Y by 1
to reflect the matrix indicies in Matlab.
\\
\\

The program written has 6 main components:
\\
Loading the image \\
Converting the image into greyscale
\\
Getting the Fourier Coefficients
\\
Shifting the zero frequency component
\\
Applying the log function to bring out any patterns in the image
\\
Reconstructing the image
\clearpage
\section{Results}
Code can be accessed here:
\\https://github.com/Soham-Deshpande/FFTBadminton

\begin{figure}[h]
    \centering
    \includegraphics[scale=1]{image1.png}
    \caption{Original image}
\end{figure}
\begin{figure}[h]
    \centering
    \includegraphics[scale=1]{image2.png}
    \caption{Grey scale image}
\end{figure}
\begin{figure}[h]
    \centering
    \includegraphics[scale=1]{image3.png}
    \caption{Fourier transform of image}
\end{figure}
\begin{figure}[h]
    \centering
    \includegraphics[scale=1]{image4.png}
    \caption{Centered fourier transform of image}
\end{figure}
\begin{figure}[h]
    \centering
    \includegraphics[scale=1]{image5.png}
    \caption{Log fourier transform of image}
\end{figure}
\begin{figure}[h]
    \centering
    \includegraphics[scale=1]{image6.png}
    \caption{Reconstructed image }
\end{figure}

\clearpage
\section{Conclusion}
\\
Beyond image processing, the fourier transform has a multitude of uses. An Application
is in modelling time series data which is very relevant to the stock market. I aim to further
my knowledge about this powerful tool and look to implement it in my own tools
\end{document}

